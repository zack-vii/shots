\documentclass[12pt]{article}

\usepackage{multirow}
\usepackage{makeidx}
\usepackage{graphicx} 
\usepackage{isodateo}
\usepackage{lastpage}
\usepackage[utf8]{inputenc}

\makeindex

\title{Shots to show}
\author{Gianluca Moro\footnote{Gianluca.Moro@unipd.it - at Max-Planck-Institut f\"ur Plasmaphysik, Greifswald, DE}}
\date{25 April 2018}

\begin{document}

\maketitle

\section{Introduction}

The project implements a web page which can give a compact overview
of the shots' statuses of a day. It is possible to:
\begin{itemize}
\item select a date;
\item show all the shots present in that date and their status;
\item check the shots to flag them as ready to be copied.
\end{itemize}

The web app is implemented in Php and uses a SQLite database as storage
for flags status (by Timo Schr\"oder).


\section{Code overview}


\subsection{Libraries}

The page uses the following libaries, stored in \texttt{web/js}:
\begin{itemize}
\item Twig (\texttt{https://twig.symfony.com/}, used in the Symfony framework, 
  but can be used standalone too): is a templating library for web pages
\item Kalendae (\texttt{https://github.com/Twipped/Kalendae}):
  is an utility do show a calendar - the usage of this library can be avoided
  using the standard HTML input date form (the solution used at the moment).
  Directory \texttt{archive} stores the used source code.
\item liblog.php: some logging functions (not used at the moment) 
\end{itemize}

\subsection{Main code}


The Php source files are in the main directory:
\begin{itemize}
\item config.php: keeps some configuration variables
\begin{description}
\item[\$ipAllowedToCopy] keeps a list of IPs allowed to check shots: 
  if the connection comes from an allowed IP that user can check shots.
  If the list is empty, everyone can check shots.
\item[\$shotsDirectory] the directory storing MDSplus shots
\item[\$mainW7XTreeName] the tree which is used to get the list of shots: 
  if a shot is present, it must be present in this directory.
\item[\$statusDefinitions] the map from status as recovered from database to
  its color, the presence (\texttt{true}) or absence (\texttt{false}) of checkbox,
  and a message shown on the web page legend.
\item[\$dbFilename] the SQLite file with the shots' database and the variable \textbf{\$dbShots}
  containing the database (as a matter of fact, logically not a configuration variable).
\item[\$listOfExpts] the list of current trees, present in configuration as it
  was thought for a ``default'' list: in the current code it is always initialized 
  to the corrent list, so its value as defined here is never used.
\item[\$logFileName] and \textbf{\$logLevel} at the moment not used
\end{description}

\item externalCmds.php: code to interface to database and filesystem
\begin{description}
\item[getListOfTrees()] get the list of all the trees, from the external command \texttt{w7x\_gettrees}
\item[getTableOfStatus(\$listOfTrees, \$listOfShots)] for each tree in the 1$^{st}$ parameter list
  and for each shot in the 2$^{nd}$ parameter's list, a query to the database is done to get the status.
  The query is: \texttt{SELECT stat FROM shotdb WHERE shot=sName AND expt=tName;}
\item[saveShot(\$tree, \$shot)] this function checks the given shot: the real action is
  to set the status to ``2'' if is was ``1''. The query to set the status is: \texttt{UPDATE shotdb SET stat=2 WHERE shot=sName AND expt=tName;}.
\item[getListOfShots(\$dirName, \$shotDate)] get the list of the shots in a given date.
\end{description}

\item index.php: the main function is \texttt{buildHome}, which show a form to select the date,
  recover the list of shots for that day, recover all the statuses for the found shots and
  call the Twig template rendering with all the information.

  If there is a request to check (the user selected some shots and submitted them), then
  those shots are checked in the database (only if the IP is allowed)
\end{itemize}


\subsection{The html interface}

The files are in the \texttt{web} directory. The \texttt{web/js}
contains the static files for web (images and javascript libraries), while
the base \texttt{web} contains 2 Twig templates (extension phtml):
\begin{description}
\item[template.phtml] a template loading needed Javascript libraries and
  with the header layout of the page.
\item[home.phtml] the main body, with  different sections:
\begin{itemize}
\item a block which shows messages (green border) or errors (red border) if present.
\item the date selection form: here the code using ``Kalendae'' is commented.
\item a legend of the meaning of the colours
\item the main table, showing the info (color) of the statuses and checkboxes
\item the submit button if something has been checked
\end{itemize}
\end{description}



\subsection{Other files}
\begin{itemize}
\item phpinfo.php: shows Php configuration
\item archive: source of kalendae library
\item testauth: An authentication test page with kerberos
\item testShots: a local simulation of a tree structure
\end{itemize}


\section{Information on the infrastructure}

\begin{itemize}

\item The source is in \texttt{mds-data-1:/home/gimo/shots}, and can be tested
  by running in that directory the command \texttt{php -S 10.44.4.11:4000} and
accessing the web page by \texttt{http://10.44.4.11:4000/index.php}.
The present documentation is in \texttt{mds-data-1:/home/gimo/shots-doc}.


\item Server with MDSplus data:
 \texttt{mds-control},
 \texttt{mds-data-1},
 \texttt{mds-data-2}.

\item The trees are stored in:
\texttt{/w7x/vault/}\textit{treename}, with the following filename format:
\texttt{w7x\_yymmddnnn}.

\item The status is coded as integer:
\begin{itemize}
\item $<=$0 - RED - NOT READY
\item 1  -    RED - CHECK - ready to be checked
\item 2  -    ORANGE    - checked
\item $>$2 -  GREEN - PROCESSED
\end{itemize}
\end{itemize}

\end{document}
